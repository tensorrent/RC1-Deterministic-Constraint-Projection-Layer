\documentclass{article}

\usepackage{amsmath, amssymb}
\usepackage{geometry}
\usepackage{hyperref}
\geometry{margin=1in}

\title{RC1: A Deterministic Structural Constraint Projection Framework for Text Evaluation}
\author{Brad Wallace}
\date{March 2026}

\begin{document}

\maketitle

\begin{abstract}
We introduce RC1, a deterministic rule-based framework for projecting textual outputs into a structural compliance space defined by seven lexical constraint operators. RC1 computes a normalized compliance score, violation taxonomy, and gate classification under a fixed versioned specification. The system is idempotent, dependency-free, and fully auditable. A 200-document clean technical corpus evaluation yields a measured false positive rate of 0\%, with Wilson 95\% confidence interval $< 1.8\%$. RC1 does not evaluate semantic truth or factual correctness; it enforces structural discipline only.
\end{abstract}

\section{Introduction}

Large language model outputs frequently exhibit structural inflation, abstraction escalation, prescriptive drift, and rhetorical amplification. While semantic evaluation frameworks exist, fewer systems target structural discipline under deterministic rules.

RC1 addresses this gap by defining a fixed constraint projection operator over tokenized text.

\section{Formal Framework}

Let $y$ denote a text sequence.

RC1 defines a projection operator:

\[
P(y) = \{ S, V, T, G \}
\]

Where:

\begin{itemize}
\item $V$ = total violation severity
\item $S = 1 - \frac{V}{14}$ = compliance score
\item $T$ = taxonomy vector
\item $G$ = gate classification
\end{itemize}

Maximum possible severity:

\[
V_{\max} = 14
\]

\section{Constraint Operators}

RC1-2026-03-25 includes seven fixed operators:

\begin{enumerate}
\item Undissolved Metaphor (H2)
\item Absolute Claim Without Scope (ABS)
\item Intent Without Mechanism (INTENT)
\item Abstraction Escalation (ESC)
\item Rephrasing Loop (LOOP)
\item Ungrounded Prescriptive (PRESC)
\item Self-Reference Claim (SELF)
\end{enumerate}

Each operator applies lexical window-based pattern detection. Operators are implemented as pure functions: $C_i : y \rightarrow \{0, 1, 2\}$.

\section{Determinism and Idempotence}

RC1 guarantees:

\begin{itemize}
\item No stochastic processes
\item No model inference
\item Pure standard library implementation (Python \texttt{re}, \texttt{string}, \texttt{typing})
\item Identical input yields identical output
\end{itemize}

\section{False Positive Study}

A corpus of 200 purely technical documents was evaluated. Sources: Python math stdlib, RFC~793, RFC~2616, NumPy ndarray reference, Euclidean algorithm documentation.

Results:

\[
FP = 0\%
\]

Total severity across corpus: 32. Mean $V$: 0.16. StdDev $V$: 0.54.

Wilson 95\% CI:

\[
FP_{true} < 1.8\%
\]

\section{Limitations}

RC1:

\begin{itemize}
\item Is lexical and rule-based
\item Does not detect hallucinations
\item Does not evaluate factual correctness
\item May miss subtle semantic violations
\item Is sensitive to phrasing variations
\end{itemize}

\section{Governance}

Version RC1-2026-03-25 is immutable. Operator additions require new version designation (RC2+). Threshold changes require new version. No silent modifications are permitted.

\section{Conclusion}

RC1 provides a deterministic structural discipline layer for textual evaluation. It complements semantic and safety evaluation systems without attempting to replace them. Its strength lies in scope restraint, reproducibility, and immutability.

\end{document}
